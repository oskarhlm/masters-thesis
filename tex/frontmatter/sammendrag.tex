\begin{otherlanguage}{norsk}

    \section*{Sammendrag}

\end{otherlanguage}

\begin{comment}
Husk at hvis du er en norsk student og skriver masteren din på engelsk, så \textit{må\/} du lage et sammendrag på norsk.
Bruk ikke Google Translate eller lignende, uten skriv teksten direkte på norsk.
Sammendraget trenger absolutt ikke å være identisk ord-for-ord med abstract, men skal selvsagt ha i prinsipp samme innehold, på semantisk nivå.

(If you are a non-Norwegian student, it is not obligatory to include an abstract in Norwegian.)

For those who write a Norwegian summary, whatever you do, do \textit{not\/} just directly translate the English abstract.
It might be tempting to think that the Norwegian summary is something you can do on the fly --- maybe assuming that nobody will read it.
However, in fact the opposite might be true: it is very likely that it will be read by the people you most want to make a good impression on,
such as your friends, family, and future employers.
\end{comment}

Nyere teknologiske utviklinger innenfor språkprosessering har ledet til utviklingen av kraftige store språkmodeller, for eksempel de som driver ChatGPT --- en \acrshort{acr:ai}-basert chattegrensesnitt skapt av OpenAI. Store språkmodeller har vist seg å være allsidige, og denne masteroppgavens vil teste denne allsidigheten ved utnytte språkmodellenes evner til logisk resonnering og kodegenerering til å utvikle en \acrshort{acr:gis}-applikasjon som brukeren interagerer med gjennom et chattegrensesnitt ved bruk av naturlig språk. Applikasjonen har fått navnet \textit{GeoGPT} og inneholder tre forskjellige agenttyper som kan utføre vanlige \acrshort{acr:gis}-analyser på data fra \gls{acr:osm} med minimal hjelp fra brukeren. Ved å utnytte såkalt \textit{function calling}, er agentene i stand til å bestille kall av forhåndsdefinerte funksjoner/verktøy med parametere spesifisert av språkmodellen selv. Agentene varierer i hvilket sett med verktøy de har blitt gitt, og i måten de får tilgang til dataene på --- som er identisk på tvers av agentene. Den ene agenten har tilgang til datagrunnlaget gjennom en PostGIS-database, en annen gjennom et \acrshort{acr:ogc} \acrshort{acr:api} Features-endepunkt, og den tredje ved å ha direkte tilgang til shapefiler lagret lokalt i miljøet den kjører i. Resultat fra eksperimenter viser at PostGIS-agenten løser de fleste oppgavene korrekt, har kortest mediantid per oppgave, og er billigst hva gjelder bruk av \textit{tokens}. Resultater fra et eksperiment utført for å evaluere betydningen av den innledende meldingen fra brukeren av GeoGPT viser at et mer detaljert trinn-for-trinn-melding --- som ligner det fra en person med stor erfaring med \acrshort{acr:gis} kunne skrevet --- betydelig forbedrer sjansen for at GeoGPT produserer et vellykket svar. Samlet sett viser arbeidet i denne masteroppgaven at et språkmodellbasert \acrshort{acr:gis} som GeoGPT kan øke produktivitet ved å løse vanlige \acrshort{acr:gis}-oppgaver basert på \enquote{spørringer} formulert ved naturlig språk, men at \acrshort{acr:gis}-ekspertise fortsatt er nødvendig ettersom oppgavene blir mer utfordrende.