\begin{otherlanguage}{norsk}

    \section*{Sammendrag}

\end{otherlanguage}

\begin{comment}
Husk at hvis du er en norsk student og skriver masteren din på engelsk, så \textit{må\/} du lage et sammendrag på norsk.
Bruk ikke Google Translate eller lignende, uten skriv teksten direkte på norsk.
Sammendraget trenger absolutt ikke å være identisk ord-for-ord med abstract, men skal selvsagt ha i prinsipp samme innehold, på semantisk nivå.

(If you are a non-Norwegian student, it is not obligatory to include an abstract in Norwegian.)

For those who write a Norwegian summary, whatever you do, do \textit{not\/} just directly translate the English abstract.
It might be tempting to think that the Norwegian summary is something you can do on the fly --- maybe assuming that nobody will read it.
However, in fact the opposite might be true: it is very likely that it will be read by the people you most want to make a good impression on,
such as your friends, family, and future employers.
\end{comment}

Nyvinninger innenfor språkprosessering har ført til fremveksten av meget kraftige og store språkmodeller, for eksempel modellene som ligger bak ChatGPT --- en \acrshort{acr:ai}-basert chat-applikasjon utviklet av OpenAI. Slike \textit{generative} språkmodeller har vist seg å være allsidige, og denne masteroppgaven vil utnytte språkmodellenes evner til logisk resonnering og kodegenerering, til å utvikle en \acrshort{acr:gis}-applikasjon som kan løse \acrshort{acr:gis}-oppgaver basert kun på en brukers tekstlige\todo{forslag til alternative formuleringer tas imot med åpne armer} problemformulering. Applikasjonen har fått navnet \textit{GeoGPT}, og tilbyr tre forskjellige agenttyper som gjennom eksperimenter har vist at de kan utføre en rekke ulike \acrshort{acr:gis}-analyser på \gls{acr:osm}-data, uten nevneverdig hjelp fra brukeren. GeoGPT utnytter \textit{function calling}, som i prinsippet gir agentene muligheten til å ta i bruk eksterne verktøy. Agentene har blitt utdelt forskjellige verktøy, og har også forskjellige måter å aksessere \gls{acr:osm}-dataene på. Den ene agenten bruker en PostGIS-database, en annen bruker en \acrshort{acr:ogc} \acrshort{acr:api} Features-server for å laste ned GeoJSON over \acrshort{acr:http}, og den tredje har direkte tilgang til shapefiler lagret lokalt i miljøet som GeoGPT kjører i. Eksperimentelle resultater, basert på en ny \acrshort{acr:gis}-benchmark, viser at \acrshort{acr:sql}/PostGIS-agenten gir korrekt svar på flest oppgaver, i tillegg til å være raskest og billigst. Resultater fra et eksperiment utført for å evaluere betydningen av kvaliteten til problemformuleringen til brukeren, viser at det å gi GeoGPT sekvensielle trinn for å løse problemet, forbedrer sjansen betraktelig for at den produserer riktig svar. Samlet sett viser arbeidet i denne masteroppgaven at et språkmodellbasert \acrshort{acr:gis}, slik som GeoGPT, kan løse mange vanlige \acrshort{acr:gis}-oppgaver basert kun på spørringer formulert ved naturlig språk, men også at \acrshort{acr:gis}-ekspertise blir viktigere og viktigere etter hvert som oppgavene blir mer utfordrende.


