\begin{longtable}{lp{1.8cm}p{7cm}}
\caption{Questions for experience level tests} \label{tbl:questions-prompt-levels} \\
\toprule
\textbf{Query} \textbf{ID} & \textbf{Level} & \textbf{Formulation} \\
\midrule
\endfirsthead
\caption[]{Questions for experience level tests} \\
\toprule
\textbf{Query} \textbf{ID} & \textbf{Level} & \textbf{Formulation} \\
\midrule
\endhead
\midrule
\multicolumn{3}{r}{Continued on next page} \\
\midrule
\endfoot
\bottomrule
\endlastfoot
oslo\_bergen\_geodesic & novice & Please plot the shortest flight path on a map between Oslo and Bergen's airports. \\
oslo\_bergen\_geodesic & expert & 1. Get info on all potentially relevant datasets. The airports are possibly stored as polygons in the available data. 
2. Filter those datasets for airports and list the names of the airports. 
3. Retrieve the geographic coordinates for Oslo Gardermoen Airport (OSL) and Bergen Flesland Airport (BGO) by filtering on the names you found. 
4. If no name was found, try a different dataset and go back to step 2. 
5. Utilize available tools to draw a geodesic curve that represents the shortest path on the earth's surface between these two points.
6. Present the findings with a map highlighting the largest county. \\
oslo\_roads\_gte\_70\_kmh & novice & Draw roads in Oslo where you can drive at least 70.  \\
oslo\_roads\_gte\_70\_kmh & expert & 1. Retrieve an outline of Oslo. 
2. Calculate max/min lat/lon values for this bounding box, and use it to retrieve a subset of the road data. 
3. Select road segments within the outline from step 1 that have a max speed >= 70.  
4. Present the findings with a map highlighting the selected roads. \\
num\_trees\_munkegata & novice & Could you count how many trees there are on Munkegata street in Trondheim? \\
num\_trees\_munkegata & expert & 1. List all datasets that could possibly include trees. 
2. Find the correct feature class and filter the relevant dataset to access tree data for Trondheim. Use a bounding box to reduce the number of trees to analyse. 
4. Fetch road data for Munkegata. Use a bounding box for Trondheim in case there are streets elsewhere named Munkegata. 
5. Convert both datasets to a suitable metric CRS and add a 20-meter buffer around the road data. 
6. Find all trees that lie within this buffer and count them. 
7. Present the findings with a map highlighting the roads and the trees.  \\
\end{longtable}
