\chapter{Future Work}
\label{cha:future-work}

\begin{comment}
Consider where you would like to extend or improve this work, or how somebody else could continue it.
These extensions might either be continuing the ongoing direction or taking a side direction that became obvious during the work.
Further, possible solutions to limitations in the work conducted, highlighted in Section~\ref{sec:discussion} may be presented.

Note that in the Specialisation Project Report, the Future Work section will be a key part of your plan for the novel work to be carried out in the next semester,
while in the Master's Thesis, the Future Work section rather will point to issues that others might be interested in addressing.
This can include options and alternatives that you did not try out yourself, or potential improvements and extensions to your experiments or system.
\end{comment}

This chapter will present challenges that are reserved for future work. \Autosectionref{sec:no-clear-answer} addresses a limitation with the experiments conducted in this thesis, and emphasizes the need for experiments that investigate the ability of a system like GeoGPT to answer questions where there are no concrete answers, and where ethical considerations and assumptions must be made. \Autosectionref{sec:comparing-different-models} highlights the need for testing different models for for use in \acrshort{acr:llm}-based \acrshortpl{acr:gis}. \Autosectionref{sec:automated-data-discovery} discusses automated data discovery. Currently, to use GeoGPT, the user has to provide it with data through, for instance, a geospatial database, but not all users will have the technical know-how required to set up such a database themselves.

\section{Measuring Ability to Answer Questions That Have No Clear Answer}
\label{sec:no-clear-answer}

The experiments conducted in this thesis focused on the system's ability to answer questions that have concrete and \textit{correct} answers. However, GeoGPT's ability to answer questions of a subjective nature --- questions that have no \textit{one} correct answer --- was not tested. For instance: what would happen if we asked GeoGPT to find a suitable route from A to B that is as \textit{safe} as possible? How would it interpret such a request? Would it only take into account the speed limit and road type? Would it be able to assess socio-economic aspects of areas that the route crosses, for instance avoiding \enquote{bad neighbourhoods} at nighttime? Would it decide to incorporate weather forecasts into the analysis? Future research should find methods to measure the ability of \acrshort{acr:llm}-based \acrshort{acr:gis} agents to provide appropriate and safe answers to such questions.

\section{Comparing Different Models}
\label{sec:comparing-different-models}

GeoGPT is based around OpenAI's \acrshort{acr:gpt}-4 Turbo model but, as \autoref{subsec:sota-decoder-only-llms} showed, there are numerous competitors. Future research should look into the possibility of swapping out \acrshort{acr:gpt}-4 Turbo with other models, first and foremost those with good function calling abilities, as this necessary for GeoGPT to work as intended. A benchmark comparing results for different models would be a natural way of building upon the results of this thesis.

Future research should especially look into the viability of using open-source models. In a report interviewing 500 companies on their \acrshort{acr:llm} adoption, 46 percent stated their preference for open-source models going into 2024 \citep{wangsarah16ChangesWay2024}. \textit{Control} and \textit{customizability} turns out to be the two most important factors into enterprise's open-source appeal, allowing for increased control over proprietary data and ability to effectively fine-tune models, respectively.

\section{Automated Data Discovery}
\label{sec:automated-data-discovery}

The experiments in this thesis were based upon a pre-existing collection of geospatial datasets that were made available to GeoGPT in different ways. GeoGPT is currently reliant on the user providing data either as a geospatial database, through an \acrshort{acr:ogc} \acrshort{acr:api} Features server, or as a collection of geospatial files that can be manipulated through Python code. A fully autonomous \acrshort{acr:gis} agent should, however, be able to search the web for suitable datasets. For instance, if a user wishes to perform a noise analysis for a particular location in Norway, the agent should be able to search a website like Geonorge\footnote{\url{https://www.geonorge.no/} (last visited on 6th June 2024)} for datasets related to noise (firing ranges, roads, etc.), download these, and then perform the analysis. Initial test were conducted with Geonorge in this thesis to see if this was possible, but the results were inconsistent.
