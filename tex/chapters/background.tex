\chapter{Background Theory}
%OR: \chapter{Tools and Methods}
\label{cha:background-theory}

\begin{comment}
The background theory depth and breadth depend on the depth needed to understand your project
in the different disciplines that your project crosses.
It is not a place to just write about everything you know that is vaguely connected to your project.
The theory is here to help the readers that do not know the theoretical basis of your work so that they
can gain sufficient understanding to understand your contributions --- and also for yourself to show that
you have understood the underlying theory and are aware of the methods used in the field.
In particular, the theory section provides
an opportunity to introduce terminology that can later be used without disturbing the text with a definition.
In some cases it will be more appropriate to have a separate section for different theories (or even separate chapters).
However, be careful so that you do not end up with too short sections.
Subsections may also be used to separate different background theories.

Be aware that ``background'' is a general term that refers to everything done by somebody else,
in contrast to the ``foreground'', which is your own work.
Hence there can (and will) be several background chapters, with the background theory being one of them
--- or several of them, since it thus is quite possible to split the background theory over more than one chapter,
e.g., by having a chapter introducing the theory directly needed for the research field in question and another
chapter discussing the machine learning theory, algorithms, tools, and evaluation methods commonly used in the field.
The related work chapter is thus also part of the background, while a chapter about data might be background
(if you only use somebody else datasets), but can also be part of the foreground (if you collect and/or annotate data
yourself, or if you process or clean the data in ways that can make it part of your own contribution).

It is ok to reuse material from other texts that you have written (e.g., the specialisation project), but if you do so, that must be clearly stated in the text, together with a description of how much of the text is new, old or rewritten/edited.
Such a statement about recycling of material in the Background Theory chapter can thus come here in the chapter introduction.

\section{Writing References in the Text}
\label{sec:writing_references}

When introducing techniques or results, always reference the source.
Be careful to reference the original contributor of a technique and not just someone who happens to use the technique.%
\footnote{But always make sure that you have read the work you are citing --- if not, cite someone who has!}
For results relevant to your work,
you would want to look particularly at newer results so that you have referenced the most up-to-date work in your area.
A common rule of thumb is to at least reference the first paper introducing the issue and the paper containing the latest / state-of-the-art
results. Additional papers making substantial contributions should also be referenced, as well as of course the ones you find most interesting.
Remember to use the right verb form depending on the number of authors.

If you do not have the source handy when writing, mark in the text that a reference is needed and add it later. \todo{add reference}
Web pages are not reliable sources --- they might be there one day and removed the next; and thus should be avoided, if possible.
A verbal discussion is not a source and should normally not be referenced
(though you can reference ``personal communication'', if there are no other options).
The bulk of citations in the report will appear in Chapter~\ref{cha:related_work}.
However, you will often need to introduce some terminology and key citations already in this chapter.

You can cite a paper in the following manner (and several other versions,
see the \verb!natbib! package documentation):

\begin{enumerate}[(i)]
    \item When referring to authors, using their names in the text:\\
          \citet{Authorson;Bobsen:10} stated something rather nice.
          (using \verb!\citet!)
    \item To cite indirectly: \\
          Papers should be written nicely \citep{Authorson;Bobsen:10}
          (using \verb!\citep!)
          {\em or\/}\\
          In \citet{Authorson;Bobsen:10}, a less detailed template was presented.
    \item To just cite the authors: \\
          \citeauthor{Authorson;Bobsen:10} wrote a nice paper
          (using \verb!\citeauthor!).
    \item Or just the year: \\ \citeyear{Authorson;Bobsen:10}
          (using \verb!\citeyear!).
    \item You can even cite specific pages or chapters: \citet[p. 3]{Authorson;Bobsen:10}
          (using \verb!\citet[...]{...}!).
\end{enumerate}

You should obviously always cite your supervisor's work \citep{BenyonEA:13},
even if it is completely irrelevant \citep{Das;Gamback:13a} or very old \citep{AlshawiEA:91b}.
Digging up an even older book can also appear impressive \citep{Diderichsen:57}.
(Or? ;-)

\section{The Reference List}
\label{sec:reference_list}

In general, make sure that the references that appear in your reference list can be easily located and identified by the reader.
So include not only authors and title, but year and place of publication, the full names of conferences and workshops,
page numbers in proceedings and collections, etc.
Hyperlinks or \acrfull{acr:doi} numbers are also nice to include.
Just as in the text itself, it is important to be consistent in the reference list, so include the same type of information for all references and write it in the same way.

Check out the reference list at the end of this document for examples of how to write references in \BibTeX.
Note a particular quirk: Many \BibTeX\ styles convert uppercase letters to lowercase, unless specifically told not to.
You might thus need to ``protect'' characters that should not be converted, e.g., by writing \texttt{\{T\}witter} as in the \citet{FountaEA:18} reference.

Also, keep in mind that `et' is a word in its own right (`and'), so there is no period after it (even though there is a period after `al.', which is short for `alia', meaning `others').
Of course, when including such a reference in the text, the authors should be referred to in plural form.
So \citet{BenyonEA:13} state that life is good (not ``states'').

Many sites, such as journals and \url{dblp.org} provide the matching \BibTeX\ entry for a reference.
However, you might still need to edit the entry in order to be consistent with the rest of your references.
If you find references from sites such as \url{scholar.google.com} or \url{arXiv.org}, keep in mind that they often not are complete,
so that you might need to add information to the entry (and probably edit it as well).

Some other good sites to find state-of-the-art work:
\begin{itemize}
    \item \url{paperswithcode.com}
    \item \url{nlpprogress.com}

\end{itemize}

\textit{Lorem ipsum dolor sit amet, consectetur adipiscing elit, sed do eiusmod tempor incididunt ut labore et dolore magna aliqua. Ut enim ad minim veniam, quis nostrud exercitation ullamco laboris nisi ut aliquip ex ea commodo consequat. Duis aute irure dolor in reprehenderit in voluptate velit esse cillum dolore eu fugiat nulla pariatur. Excepteur sint occaecat cupidatat non proident, sunt in culpa qui officia deserunt mollit anim id est laborum.}

\begin{figure}[t!]
    \centering
    \includegraphics[width=0.5\columnwidth]{figs/figure1.pdf}
    \caption[Boxes and arrows are nice]{Boxes and arrows are nice (adapted from \citealp{Authorson;Bobsen:10}, reprinted with permission)}
    \label{fig:BoxesAndArrowsAreNice}
\end{figure}

\section{Introducing Figures}

\LaTeX is a bit tricky when it comes to the placement of ``flooting bodies'' such as figures and tables. It is often a good idea to let their code appear right before the header of the (sub)section in which they appear.
Note that you should anyhow always use an option for the placement (e.g., \verb|[t!]| to place it at the top of a page).

Remember that if you reproduce someone else's figures you must credit the original author --- such as
Figure~\ref{fig:BoxesAndArrowsAreNice} (adapted from \citealp{Authorson;Bobsen:10}),
as well as state that you have permission to reprint it (e.g., if it is published under a Creative Commons License,
or if you have gained explicit permission from the author).

Do not just put the figure in and leave it to the reader to try to understand what the figure is.
The figure should be included to convey a message and you need to help the reader to understand the message
intended by explaining the figure in the text.
Hence \textbf{all} figures and tables should always be referenced in the text, using the \verb!\ref! command.
It is good practice to always combine it with a non-breakable space (\verb!~!) so that there will be no newline between the term referring to it and the reference, that is, using \verb!Figure~\ref{fig:BoxesAndArrowsAreNice}!.

If a figure appears far from the text explaining it,
it is a good idea to add its page number (using the \verb!\pageref! command), so that you can refer to Figure~\ref{fig:BoxesAndArrowsAreNice} (on Page~\pageref{fig:BoxesAndArrowsAreNice}).

Also, note that you can have a longer version of the figure (and table) caption attached to the actual figure,
while using the optional first argument to \verb!\caption! to include a shorter version in the list of figures (lof) or list of tables:
\begin{quote}
    \begin{verbatim}
\caption[Shorter lof text]{Longer text appearing under the figure}
\end{verbatim}
\end{quote}

It is good practice to add a note about a missing figure in the text,
such as the completely amazing stuff that will appear in Figure~\ref{fig:AmazingFigure}.

\begin{figure}[t!]
    \centering
    \missingfigure{Here we will add an amazing figure explaining it all}
    \caption{A missing figure}
    \label{fig:AmazingFigure}
\end{figure}

In general it is good to add notes about things that you plan on writing later.
The \verb!todonotes! package is great for that kind of book-keeping, letting you write both shorter comments in the margin\todo{l8r dude} and longer comments inside the text, using the option \verb![inline]!.
\todo[inline]{There are always some more stuff that you will need to add at some later point.
    Be sure to at least make a note about it somewhere.}

\textit{Sed ut perspiciatis unde omnis iste natus error sit voluptatem accusantium doloremque laudantium, totam rem aperiam, eaque ipsa quae ab illo inventore veritatis et quasi architecto beatae vitae dicta sunt explicabo. Nemo enim ipsam voluptatem quia voluptas sit aspernatur aut odit aut fugit, sed quia consequuntur magni dolores eos qui ratione voluptatem sequi nesciunt. Neque porro quisquam est, qui dolorem ipsum quia dolor sit amet, consectetur, adipisci velit, sed quia non numquam eius modi tempora incidunt ut.}

\section{Introducing Tables in the Report}

\newcommand\emc{-~~~~}
\begin{table}[t!]
    \centering
    \caption[Example table]{Example table (F$_1$-scores); this table uses the optional shorter caption that will appear in the list of tables, so this long explanatory text will not appear in the list of tables and is only here in order to explain that to the reader.}
    \begin{tabular}{c|c|rrrrrr}
        \tabletop
        Langs                  & Source                                           & \multicolumn{1}{c}{Lang1} & \multicolumn{1}{c}{Lang2} & \multicolumn{1}{c}{Univ} & \multicolumn{1}{c}{NE} & \multicolumn{1}{c}{Mixed} & \multicolumn{1}{c}{Undef}
        \\ \tablemid
        \multirow{5}{*}{EN-HI} & FB+TW                                            & 54.22                     & 22.00                     & 19.70                    & 4.00                   & 0.05                      & 0.03                      \\
                               & FB                                               & 75.61                     & 4.17                      & 18.00                    & 2.19                   & 0.02                      & 0.01                      \\
                               & TW                                               & 22.24                     & 48.48                     & 22.42                    & 6.71                   & 0.08                      & 0.07                      \\
                               & Vyas                                             & 54.67                     & 45.27                     & 0.06                     & \emc                   & \emc                      & \emc                      \\
                               & FIRE                                             & 45.57                     & 39.87                     & 14.52                    & \emc                   & 0.04                      & \emc                      \\ \tablemid
        \multirow{2}{*}{EN-BN} & TW                                               & 55.00                     & 23.60                     & 19.04                    & 2.36                   & \emc                      & \emc                      \\
                               & FIRE                                             & 32.47                     & 67.53                     & \emc                     & \emc                   & \emc                      & \emc                      \\ \tablemid
        EN-GU                  & FIRE                                             & 5.01                      & \textbf{94.99}            & \emc                     & \emc                   & \emc                      & \emc                      \\
        \tablemid
        DU-TR                  & Nguyen                                           & 41.50                     & 36.98                     & 21.52                    & \emc                   & \emc                      & \emc                      \\ \tablemid

        EN-ES                  & \multirow{4}{*}{\rotatebox[origin=c]{90}{EMNLP}}
                               & 54.79                                            & 23.50                     & 19.35                     & 2.08                     & 0.04                   & 0.24                                                  \\
        EN-ZH                  &                                                  & 69.50                     & 13.95                     & 5.88                     & 10.60                  & 0.07                      & \emc                      \\
        EN-NE                  &                                                  & 31.14                     & 41.56                     & 24.41                    & 2.73                   & 0.08                      & 0.08                      \\
        AR-AR                  &                                                  & 66.32                     & 13.65                     & 7.29                     & 11.83                  & 0.01                      & 0.90                      \\ \tablebot
    \end{tabular}
    \label{tab:ExampleTable}
\end{table}

As you can see from Table~\ref{tab:ExampleTable}, tables are nice.
However, again, you need to discuss the contents of the table in the text.
You do not need to describe every entry, but draw the reader's attention to what is important in the table,
e.g., that 94.99 is an amazing F$_1$-score (and that probably something fishy happened there).
Use boldface, boxes, colours, arrows, etc. to mark the important parts of the table.

As can be seen in the example, elements in a table can sometimes benefit from being rotated (such as EMNLP in the `Source' column) or cover more than one row (EMNLP, as well as EN-HI and EN-BN in the `Langs' column) --- or more than one column, for that matter.

\textit{Donec non turpis nec neque egestas faucibus nec id neque. Etiam consectetur, odio vitae gravida tempus, diam velit sagittis turpis, a molestie ligula tellus at nunc. Proin dolor neque, dapibus a pellentesque a, commodo a nibh.}
\end{comment}

\begin{itshape}
    NB! Parts of the \nameref{cha:background-theory} chapter is reused material from the specialization project \citep{holmLLMsDeathGIS2023} preceding this master thesis. Below are the sections in question, together with a description of the extent to which, and how, the material is reused:

    \begin{itemize}
        \item \Autosubsectionref{subsec:attention-and-the-transformer-architecture}: Reused with minor adjustments.
        \item \Autosubsectionref{subsec:sota-llms}: \acrshort{acr:gpt} part reused without modification.
    \end{itemize}
\end{itshape}

\vspace{12pt}

\noindent \Autochapterref{cha:background-theory} will lay a theoretical basis for the work done in this master thesis, providing the user with the required understanding in order to understand the contributions of the work. \Autosectionref{sec:language-modelling} will explain the theoretical basis of the component which most modern \glspl{acr:llm} are based upon --- namely the Transformer --- and the attention mechanism within it. The section will also touch upon a new approach to language modelling called \textit{selective state space modelling}, which has yielded very promising results for small \glspl{acr:llm}.


% \section{Approaches to Language Modelling}
% \label{sec:language-modelling}

% \Autosubsectionref{subsec:attention-and-the-transformer-architecture} will explain the theoretical basis behind most modern \glspl{acr:llm}, which are based upon the attention mechanism built into the Transformer architecture. \Autosubsectionref{subsec:state-space-models} will explain modern state space-modelling approaches and why they may have a potential greater than Transformer-based models. First, however, \autoref{subsec:statistical-models-and-rnns} will delve into earlier attempts at language modelling.

% \subsection[Early Attempts: Statistical Models and Recurrent Neural Networks]{Early Attempts: Statistical Models and \acrlongpl{acr:rnn}}
% \label{subsec:statistical-models-and-rnns}

% \subsection{Statistical Models}


% \subsection[Recurrent Neural Networks]{\acrlongpl{acr:rnn}}


% \subsection{State Space Models}
% \label{subsec:state-space-models}


\section[Large Language Models]{\acrlongpl{acr:llm}}

\glspl{acr:llm} are a type of neural networks that excel at processing language. They can be developed for different \gls{acr:nlp} tasks, such a text classification, masked language modelling, and text generation. While they all have their use cases, only text generation will be relevant for this thesis.

Generative \glspl{acr:llm} are designed to take some input sequence and generate some output sequence.

This section will lay the theoretical groundwork required to gain an overview of the inner workings of \glspl{acr:llm}. \autoref{subsec:core-concepts} will inform the reader on core concepts and terminology that will be used extensively throughout this thesis when discussing \acrshortpl{acr:llm}. \autoref{subsec:attention-and-the-transformer-architecture} will explain the attention mechanism and the related Transformer architecture, the latter of which serves as the core building block in most modern \acrshortpl{acr:llm}.

% Most modern \glspl{acr:llm} use the Transformer architecture \citep{vaswaniAttentionAllYou2017} which is described in \autoref{subsec:attention-and-the-transformer-architecture}. 

\subsection{Core Concepts}
\label{subsec:core-concepts}

The \textbf{context window} of an \acrshort{acr:llm} is the range of tokens that an \acrshort{acr:llm} is able to process.

While humans understand sentences as sequences of words, \acrshortpl{acr:llm} perceive them as sequences of \textbf{tokens}. An \acrshort{acr:llm} possesses a fixed set of unique tokens in its vocabulary, from which it constructs words and sentences. Tokens can be entire words, short character sequences, or single characters. \autoref{fig:tokenization-example-sentence} illustrates how the latest \acrshort{acr:gpt} models \textit{tokenize} an English sentence. Notice how some words are deconstructed into more than one token. For example, the word \enquote{revolutionizing} is split into its root (\enquote{revolution}) and its suffix (\enquote{izing}). \autoref{fig:tokenization-revolution} shows how this process applies to other suffixes as well. Likely, the model has learned the meaning of \enquote{revolution} and elects to use different suffixes to modify its function within a sentence, as opposed to having an entirely new token for each version of the word.

\begin{figure}[htp]
    \centering
    \includegraphics[width=\textwidth]{tokens_illustration.png}
    \caption{Tokenization example for a sentence}
    \label{fig:tokenization-example-sentence}
\end{figure}

\begin{figure}[htp]
    \centering
    \includegraphics[width=0.25\textwidth]{revolution_tokens.png}
    \caption{Tokenization of \enquote{revolution} with different suffixes}
    \label{fig:tokenization-revolution}
\end{figure}

When an \acrshort{acr:llm} generates text, it does so by generating a new token based on what

\subsection{Attention and the Transformer Architecture}
\label{subsec:attention-and-the-transformer-architecture}

\cite{vaswaniAttentionAllYou2017} managed to achieve new state-of-the-art results for machine translation tasks with their introduction of the Transformer architecture. The Transformer has later been proved effective for numerous downstream tasks, and for a variety of modalities. Titling their paper \citetitle{vaswaniAttentionAllYou2017}, \citeauthor{vaswaniAttentionAllYou2017} suggest that their attention-based architecture renders network architectures like \glspl{acr:rnn} redundant, due to its superior parallelization abilities and the shorter path between combinations of position input and output sequences, making it easier to learn long-range dependencies \citep[6]{vaswaniAttentionAllYou2017}.

The Transformer employs self-attention, which enables the model to draw connections between arbitrary parts of a given sequence, bypassing the long-range dependency issue commonly found with \glspl{acr:rnn}. An attention function maps a query and a set of key-value pairs to an output, calculating the compatibility between a query and a corresponding key \citep[3]{vaswaniAttentionAllYou2017}. Looking at \citeauthor{vaswaniAttentionAllYou2017}'s proposed attention function \eqref{eq:attention}, we observe that it takes the dot product between the query $Q$ and the keys $K$, where $Q$ is the token that we want to compare all the keys to. Keys similar to $Q$ will get a higher score, i.e., be \textit{more attended to}. These differences in attention are further emphasized by applying the softmax function. The final matrix multiplication with the values $V$ (the initial embeddings of the input tokens) will yield a new embedding in which all individual tokens have some context from all other tokens. We improve the attention mechanism by multiplying queries, keys, and values with weight matrices that are learned through backpropagation. Self-attention is a special kind of attention in which queries, keys, and values are all the same sequence.


\begin{equation}
    \text{Attention}(Q, K, V) = \text{softmax}\left(\frac{QK^T}{\sqrt{d_k}}\right)V
    \label{eq:attention}
\end{equation}

Attention blocks can be found in three places in the Transformer architecture \citep[5]{vaswaniAttentionAllYou2017} (I will use machine translation from Norwegian to German as an example):

\begin{enumerate}
    \item In the encoder block to perform self-attention on the input sequence (which is in Norwegian)
    \item In the decoder block to perform self-attention on the output sequence (which is in German)
    \item In the decoder block to perform cross-attention (also known as encoder-decoder attention) where each position in the decoder attends to all positions in the encoder
\end{enumerate}

The Transformer represented a breakthrough in the field of \gls{acr:nlp}, and is the fundamental building block of modern \glspl{acr:llm}, most famous of which are the \acrshort{acr:gpt}'s.

\subsection[State-of-the-Art Decoder-Only Models]{State-of-the-Art Decoder-Only \acrlongpl{acr:llm}}
\label{subsec:sota-decoder-only-llms}

While the work of \citeauthor{vaswaniAttentionAllYou2017} is still considered perhaps the greatest breakthrough in \gls{acr:nlp}, most moderns \acrshort{acr:llm} do not apply this encoder-decoder architecture. The subsequent evolution of \acrshortpl{acr:llm} has favoured generative decoder-only models, focusing entirely on the generative component of the Transformer, hoping to create models that can produce coherent and context-aware text. The first decoder-only model was OpenAI's \acrshort{acr:gpt}-1.

\subsubsection{The GPT Family}
\label{subusubsec:gpt}

\gls{acr:gpt} is a type of \gls{acr:llm} that was introduced by OpenAI in 2018 \citep{radfordImprovingLanguageUnderstanding2018}. Specifically designed for text generation, a \acrshort{acr:gpt} is essentially a stack of Transformer \textit{decoders}. It demonstrates through its vast pre-training on unlabelled data that such unsupervised training can help a language model learn good representations, providing a significant performance boost while alleviating the dependence on supervised learning. While the original Transformer architecture as described by \cite{vaswaniAttentionAllYou2017} was intended for machine translation --- thus having encoders to learn the representation of the origin language representation of a given input sequence and decoders to learn the representation in the target language and perform cross-attention between the two --- the \acrshort{acr:gpt} is designed only to \textit{imitate} language. This is why there are no encoders to be found in the \acrshort{acr:gpt} architecture, only decoders. The model employs masked multi-head attention (running the input sequence through multiple attention heads in parallel), and is restricted to only see the last $k$ tokens --- with $k$ being the size of the context window --- and tasked to predict the next one.

Training consists of two stages: unsupervised pre-training and supervised fine-tuning. The former is used to find a good initialization point, essentially teaching the model to imitate the corpora upon which it is trained. This results in a model that will ramble on uncontrollably, just trying to elaborate upon the input sequence it's given to the best of its knowledge. This will naturally produce undefined behaviour, and it is therefore necessary to fine-tune the model on target tasks in a supervised manner. \cite[4]{radfordImprovingLanguageUnderstanding2018} explain how the model can be fine-tuned directly on tasks like text classification, but how one for other tasks needs to convert structured inputs into ordered sequences because the pre-trained model was trained on contiguous sequences of text. In the case of ChatGPT, \citeauthor{openaiIntroducingChatGPT2022} used \gls{acr:rlhf} by employing a three-step strategy: first training using a supervised policy, then using trained reward models to rank alternative completions produced by ChatGPT models, before fine-tuning the model using \gls{acr:ppo}, which is a way of training \acrshort{acr:ai} policies. This pipeline is then performed for several iterations until the model produces the desired behaviour \citep{openaiIntroducingChatGPT2022}.

\subsubsection{The Gemini Family}
\label{subsubsec:gemini}

The suite of models known as Gemini is Google's latest response OpenAI's \acrshort{acr:gpt} models. The Gemini 1.0 suite \citep{geminiteamGeminiFamilyHighly2024}, which is the first suite of Gemini models, includes three different models: Ultra, Pro, and Nano. These are listed in descending order in terms of size (number of parameters). Like most other commercial \acrshortpl{acr:llm} the models of the Gemini 1.0 suite are multimodal, supporting text, image, audio, and video. Gemini 1.0 displayed new state-of-the-art performance on most major benchmarks, but performed significantly worse on the HellaSwag benchmark (which measures a model's common-sense understanding) compared to the newest \acrshort{acr:gpt}-4 model at the time. The models scored 87.8\% and 95.3\%, respectively. Supporting a context length up to 1M tokens, Gemini 1.0 Ultra surpassed the Claude 2.1's context window of 200k with a wide margin, and with the release of Gemini 1.5 Pro came also the possibility of utilizing a context window of up to 10M tokens, though in production this number is currently 1M \citep{geminiteamGeminiUnlockingMultimodal2024, pichaiOurNextgenerationModel2024}. Furthermore, Gemini 1.5 Pro outperforms Gemini 1.0 Ultra in some capabilities despite using significantly less training compute \citep[31]{geminiteamGeminiUnlockingMultimodal2024}.

\subsubsection{The Claude Family}
\label{subsubsec:claude}

Developed at Anthropic, Claude is the third major, commercial \acrshort{acr:llm}. Anthropic is one of the actors in the \acrshort{acr:llm} market that has helped push in the direction of long-context \acrshortpl{acr:llm}, which their Claude 2 model being able to support up to 200k tokens \citep[9]{anthropicModelCardEvaluations2023}, being the best at the time of its release in November 2023. Latest in line is Claude 3, a family of \acrshortpl{acr:llm} of different sizes, largest of which is the Opus model which outperforms \acrshort{acr:gpt}-4 and Gemini 1.0 Ultra on most benchmarks \citep[6]{anthropicClaudeModelFamily2024}. Claude 3 is a family of language models, much like the Gemini family, featuring three main models: Opus, Sonnet, and Haiku. Again, these are listed in descending order in terms of number of parameters. Users can choose between Opus for the most advanced capabilities, Haiku for a fast and economical option, or Sonnet for a balance of both.

\subsubsection{Open-Source Alternatives}
\label{subsubsec:open-source-llms}

OpenAI's \acrshort{acr:gpt} models, Google's Gemini models, and Anthropic's Claude models are all commercial and closed-source. This prevents developers from downloading these models and making custom improvements to them through fine-tuning (see \autoref{subsec:fine-tuning-prompt-engineering}). For these reasons, a number of open-source \acrshortpl{acr:llm} have joined the scene.

The \textbf{Llama} family of \acrshortpl{acr:llm} from Meta AI is perhaps the most famous open-source option to the commercial, closed-source \acrshortpl{acr:llm}. At the time of writing, the last in line is the Meta Llama 3 model \citep{metaaiIntroducingMetaLlama2024} which comes in two size: 8B and 70B parameters. Both models display state-of-the-art performance on most major benchmarks compared to comparable open-source alternatives, and 70B model even surpasses closed-source models like Gemini 1.5 Pro and Claude 3 Sonnet on certain benchmarks.

\textbf{Mistral AI} is one of the most prominent actors in the world of open-source \acrshortpl{acr:llm}. Their debut model, Mistral 7B, outperformed Llama 13B (which was the best open-source \acrshort{acr:llm} at the time) across all the benchmarks they evaluated \citep{jiangMistral7B2023}. Mistral AI has also gained fame for their \gls{acr:smoe} architecture, which was introduced with the Mixtral 8x7B model \citep{jiangMixtralExperts2024}. It shares architecture with Mistral 7B, but is each layer of the model is composed of 8 feed-forward blocks. Using a router at each layer, it is able to use only 13B out of a total of 47B parameters during inference, keeping cost and latency low.

Along with their Gemini models, Google released a family of open-source models called \textbf{Gemma}, which are based on the same research conducted for their Gemini models \citep{gemmateamGemmaOpenModels2024}. Gemma comes in two sizes: 2B and 7B parameters. At its release, the Gemma 7B it surpassed Llama 2 13B and Mistral 7B \acrshortpl{acr:llm} in 11 out of 18 benchmarks. Note, however, that Llama 3 8B has improved upon its predecessor and now performs better than Gemma 7B overall.

\subsection{Fine-Tuning and Prompt Engineering}
\label{subsec:fine-tuning-prompt-engineering}

\subsection[Function Calling LLMs]{Function Calling \acrshortpl{acr:llm}}
\label{subsec:function-calling}

\textit{Function calling} --- first introduced by OpenAI \citep{eletiFunctionCallingOther2023} --- allows developers to provide function definitions to an \gls{acr:llm} and have said \gls{acr:llm} output a \acrshort{acr:json} object containing the name of one or more of the functions provided, as well as suitable arguments to these. Made possible through fine-tuning models to detect when functions should be calling, function calling makes it possible to give an \gls{acr:llm} \textit{hooks} into the real world, and provides a more reliable way for developers to integrate \glspl{acr:llm} into applications.

Possible use cases include using functions provide correct and up-to-date information that would otherwise require extensive training and fine-tuning. Having the \gls{acr:llm} use function calling for information retrieval also make them more transparent, making it possible to trace a claim back to its source, something that is normally a difficult feat with \gls{acr:llm}. Another use case might be code execution. One could imagine a rather simple function \texttt{execute\_python\_code(code: string) -> string} that takes Python code as a string and returns the standard output that results from executing that code. This is likely the principle behind products like OpenAI's Data Analysis mode (previously Code Interpreter), in which ChatGPT functions as a code executing agent that can generate, execute, and self-correct its own code. Similar functions could be constructed for \acrshort{acr:sql}, making it possible for \glspl{acr:llm} to work against relational databases. As \cite{eletiFunctionCallingOther2023} describes, function calling can also be used to extract structured data from text.



\section{LangChain}
\label{sec:langchain}

LangChain \citep{langchainaiLangchainaiLangchain2022} is an open-source project that provides tooling which simplifies the way developers interface with \glspl{acr:llm}. This tooling includes composable tools and integrations that can be used to build prompts for \acrshortpl{acr:llm}, as well as off-the-shelf chains that perform higher level tasks. Chains are \glspl{acr:dag} --- or sequences of runnables --- that take an input and produces and output. A runnable can be a prompt template with template literals that are substituted with values that are passed into the runnable. The output is the template with the template literals filled in. This output can then be chained into an \acrshort{acr:llm} runnable calls a language model using the prompt template. The output from the \acrshort{acr:llm} runnable could then be passed into an output parser, e.g. a \acrshort{acr:json} parser, that ensures that the chain outputs a \acrshort{acr:json} object. Such chains are the buildings blocks that make up LangChain.

Common use cases for LangChain are:

\begin{itemize}
    \item Building chatbots for question answering that use semantic retrieval from document store
    \item Creating agents with access to external tools be leveraging function calling (see \autoref{subsec:function-calling})
    \item Creating code executing agents for Python, \acrshort{acr:sql}, or other programming languages
\end{itemize}

In January 2024, LangChain AI rolled out a new framework called LangGraph which builds on top of the LangChain ecosystem. While the chains commonly found in LangChain are good for \gls{acr:dag} workflow, they are not suited to creating cyclic graphs. LangGraph can be used to add cycles to \acrshort{acr:llm} applications, which are important for agent-like behaviours \citep{langchainaiLangchainaiLanggraph2024}. A graph in LangGraph is a set of nodes that pass some state around, state that can be modified by each node. The nodes are connected together by edges that define what node can succeed another node. These edges can also be conditional, which routes execution to a given node based on the output from a function giving the current state. This allows for complex logic and simplifies implementation of advanced agent patterns, some of which are discussed in \autoref{sec:agent-patterns}.


\section{Geospatial Databases and Data Catalogues}
\label{sec:geo-dbs-and-data-catalogues}

This section will discuss the geospatial technologies that were used or considered for use in this master's thesis.

\subsection{PostGIS}
\label{subsec:postgis}

PostGIS \citep{PostGIS2001} is an open-source extension for the PostgreSQL \acrshort{acr:dbms}. By adding the PostGIS extension one adds support for storing, indexing, and querying geospatial data. Data can be stored in both two and three dimensions, and they can have types like points, lines, polygons. These types can be stored along with a spatial index which can significantly reduce search time for these geometries. GiST (Generalized Search Tree)\footnote{\url{https://en.wikipedia.org/wiki/GiST}} is commonly used in PostgreSQL/PostGIS to take advantage of various tree-based search algorithms that are developed to retrieve spatial features quickly.

PostGIS also comes with a plethora of spatial database functions that to analyse and process geospatial data. These are prefixed with \texttt{ST\_}, and some examples are \texttt{ST\_DWITHIN}, \texttt{ST\_BUFFER}, and \texttt{ST\_TRANSFORM}. \autoref{code:postgis-example} displays a typical query for retrieving building outlines within a bounding box specified in WGS 84 latitudes and longitudes.

\begin{lstlisting}[
    language=SQL,
    caption=PostGIS example code for retrieving building outlines within a specified bounding box,
    label=code:postgis-example
]
SELECT * 
FROM osm_buildings_polygons 
WHERE type = 'house' 
  AND ST_Intersects(geom, ST_MakeEnvelope(min_lon, min_lat, max_lon, max_lat, 4326));
\end{lstlisting}

\subsection[OGC API Features]{\acrshort{acr:ogc} \acrshort{acr:api} Features}
\label{subsec:ogc-api-features}

\acrshort{acr:ogc} \acrshort{acr:api} Features is an \acrshort{acr:api} specification that defines modular \acrshort{acr:api} building blocks for interacting with features, real-world objects \citep{opengeospatialconsortiumOGCAPIFeatures2022}. These building blocks include blocks for creating, modifying, and querying features on the Web. A typical implementation of \acrshort{acr:ogc} \acrshort{acr:api} Features implements these building blocks for \acrshort{acr:html}, GeoJSON, and \acrshort{acr:gml}. These are called \textit{requirement classes}, though none of them are strictly required. The \acrshort{acr:html} requirement class gives the user of the \acrshort{acr:api} a visualization of the features, whereas the GeoJSON and \acrshort{acr:gml} requirements classes are typically meant for use in other applications.

\begin{figure}[h]
    \centering
    \includegraphics[width=\textwidth]{ogc_api_features.png}
    \caption{Collections, items, and features in \acrshort{acr:ogc} \acrshort{acr:api} Features specification. Retrieved from \url{https://features.developer.ogc.org/} on April 29, 2024.}
    \label{fig:oaf-collections-items-features}
\end{figure}

The development of the Features standard is divided into several parts that are meant to build on top of one another. Here are the four parts that are listed on \acrshort{acr:ogc}'s site about the standard \footnote{\url{https://ogcapi.ogc.org/features/}}:

\begin{itemize}
    \item Features - Part 1: Core\footnote{\url{https://docs.opengeospatial.org/is/17-069r4/17-069r4.html}}
    \item Features - Part 2: Coordinate Reference Systems by Reference \footnote{\url{https://docs.ogc.org/is/18-058r1/18-058r1.html}}
    \item Features - Part 3: Filtering\footnote{\url{https://docs.ogc.org/DRAFTS/19-079r1.html}}
    \item Features - Part 4: Create, Replace, Update and Delete\footnote{\url{https://docs.ogc.org/DRAFTS/20-002.html}}
    \item Features - Part 5: Schemas\footnote{\url{https://docs.ogc.org/DRAFTS/23-058r1.html}}
\end{itemize}

Part 1 specifies core capabilities that are described in the first paragraph of this section, while parts 2-4 spcify additional capabilities. Part 2 allows retrieval of features in \glspl{acr:crs} different to the default WGS 84 reference system. Part 3 enables filtering of features using \gls{acr:cql}. \gls{acr:cql} is a language similar to \acrshort{acr:sql}. This allows for filtering of collections, so that users of the \acrshort{acr:api} can retrieve only a subset of a given collection. Below are two examples of \gls{acr:cql} queries:

\begin{lstlisting}[
    language=SQL,
    caption=\acrshort{acr:cql} examples,
    label=cql-examples
]
\\ Example 1
county in ('Akershus', 'Buskerud', 'Ostfold')

\\ Example 2
DWITHIN(the\_geom, Point(63.4265, 10.3960), 1, kilometers)
\end{lstlisting}

Part 4 defines how an \acrshort{acr:api} following the specification handle addition,  replacement, modification, and/or removal for a collection. Part 5 describes how features can be described by a logical schema and how these are published. Furthermore, several proposed extensions, such as the Search extension which could allow for multi-collection queries, or the Geometry Simplification extension which proposes the use of simplification algorithms for retrieving simplified versions of a collection, have been created.\footnote{\url{https://github.com/opengeospatial/ogcapi-features/tree/master/proposals}}


\subsection[SpatioTemporal Asset Catalogs]{\acrlong{acr:stac}}
\label{subsec:stac}

The \gls{acr:stac} specification\footnote{\url{https://stacspec.org/en}} is closely related to \acrshort{acr:ogc} \acrshort{acr:api} Features, and as \citeauthor{holmesSpatioTemporalAssetCatalogs2021a} --- former board member  of \acrlong{acr:ogc} --- stated in a blog post that \enquote{\acrshort{acr:stac} \acrshort{acr:api} implements and extends the \acrshort{acr:ogc} \acrshort{acr:api} — Features standard, and our shared goal is for \acrshort{acr:stac} \acrshort{acr:api} to become a full \acrshort{acr:ogc} standard} \citep{holmesSpatioTemporalAssetCatalogs2021a}. The main difference to the \acrshort{acr:ogc} \acrshort{acr:api} Features specification is its requirement that all items/features should have a temporal component, thus making it \textit{spatiotemporal}. The vision for the \acrshort{acr:stac} specification is to minimize the need for developing new code with the release of each new dataset or \acrshort{acr:api}, by using a standard that does not change.







\glsresetall
